\documentclass[a4paper,12pt]{article}
\usepackage[a4paper]{geometry}
\geometry{top=1.0in, bottom=1.0in, left=0.1in, right=0.1in}
\setlength{\textwidth}{150mm}
\setlength{\oddsidemargin}{0.1cm}
\setlength{\evensidemargin}{0.1cm}
\setlength{\marginparsep}{0.1cm}
\setlength{\marginparwidth}{0.1cm}
\setlength{\marginparpush}{0.1cm}

\usepackage{minted}

%\usepackage[italian]{babel}
\usepackage{graphicx}
\usepackage[intlimits]{amsmath}
\usepackage{amsfonts}
\usepackage{listings}
\addtolength{\oddsidemargin}{-.0in}
\addtolength{\evensidemargin}{-.0in}

\usepackage{color}

\begin{document}

\begin{center}
 \huge \bfseries {Neilos}
\\[0.5cm]
\end{center}
\tableofcontents
\newpage
\section{Introduction}
Neilos is a content managent system based almost completely on javascript. It is designed to be lightweight and fast and to use ajax for content loading. Furthermore, it is possible to use mysql, xml or both technologies to store content datas. Building a website with Neilos should be as simple as writing a basic xml file.
\section{Workflow of Neilos}
The first thing Neilos does is loading resources/xml/config.xml, which is the main configuration of the site. The main config usually (but not mandatory) loads the css styles, add the layout divs (see Structure section), and sets global variables, like default target or default animation. Thereafter, the main config loads other external files, which can add articles, comments, html, php code or load in turn other files. Every external file must be loaded with an $entryid$ parameter that tells Neilos which part of the file (entry) is to be added. \\
\includegraphics[scale=0.7]{Diagramma_neilos.png} 
\subsection{Menu links}
A typical menu link points to index.html\#page.xml\ \ .Neilos will load automatically page.xml and will add it to the site with ajax. The file must be located in resources/content/
\section{Entries Anatomy}
Every file is loaded with an $entryid$ parameter. Neilos search for the entry with the id supplied. For php files, the entry must be enclosed in at least one other element. For xml files, the entry can be the ancestor of the document.\\
An entry consists of 3 main sections: config, title and content. There can be also subentries.\\
Content and title should consist of pure html, since they will be dumped to the DOM.\\
\footnotesize
\begin{minted}{xml}
<entry id='entryid'>
  <config>
    ...
  </config>
  <title>
    ...
  </title>
  <content>
    ...
  </content>


  <!-- Subentries -->
  <entry id='..'></entry>
  <entry id='..'></entry>
</entry>
\end{minted}
\normalsize
\subsection{DOM}
By default, every entry is added to the DOM with a main div ($\#id\_entry$) and 3 subdivs: .title, .content and .comments.\\
This behaviour can eventually be changed with some options (see below). If both $<title>$ and $<content>$ do not exist, the entry is not added to the DOM.
\subsection{Config}
The config tag is useful to load additionals xml,css,php files or to set useful settings.\\
Almost all options are inheritable to subentries. Available options are:
\begin{list}{-}{}
\item \begin{minted}{xml}
<css id='css_id'>       
      \end{minted}
Load an external css file. The id is used to identify the css in order to remove or replace it afterwards.
\item \begin{minted}{xml}
<author>m3l7</author>
<date>2011-09-08</date>
      \end{minted}
Specify the date and author of the entry. They will be added to the DOM in the .title before everything else.
\item \begin{minted}{xml}
<structure_div>div_name</structure_div>       
      \end{minted}
Add a div to the DOM. (See Structure section)
\item \begin{minted}{xml}
<target>target</target>       
      \end{minted}
specify where the entry should be added in the DOM (jquery selector).
\item \begin{minted}{xml}
<animation speed='fast' speedshow='normal' speedhide='slow'
 type='fade/slide/hide'>enabled</animation>
      \end{minted}
Select the type and speed of the animation. If speedshow/speedhide is set, it will be used instead of speed for showing/hiding the entry.
The speed can have the format $\#n$, where n is the animation time in milliseconds.
\item \begin{minted}{xml}
<home>#home.xml</home>
      \end{minted}
Set the default home page. See links section for informations about links meaning.
\item \begin{minted}{xml}
<display entries="showfirst/show/hide"></display>
\end{minted}
Choose if content (and comments) should be visible or not. If showfirst is set, the entry is visible only if it's the first child.\\
Default: show
\item \begin{minted}{xml}
<load_file mode='loadwhenshown' 
entryid="menu">resources/xml/menu.xml</load_file>
      \end{minted}
Load an external file. Neilos will add the content inside the jquery selector tag.\\
If mode='loadwhenshown' is set, the file is loaded only when the entry is toggled.
\item \begin{minted}{xml}
<clear>false</clear>
\end{minted}
If true, the target will be cleared before adding the entry. Not inheritable.
\item \begin{minted}{xml}
<skipcontent>true</skipcontent>
\end{minted}
If true, title and content will be ignored.
\item \begin{minted}{xml}
<skipsubentries>true</skipsubentries>
\end{minted}
If true, subentries will be ignored.
\item \begin{minted}{xml}
<type>notitle</type>
\end{minted}
If set, title will be ignored.
\end{list}
\normalsize
\section{Variables}
Neilos has an internal parser which substitutes some special strings. At this stage, it's very limited.
\begin{list}{-}{}
\item \begin{minted}{javascript}
_$version
\end{minted}
show Neilos version
\end{list}
\normalsize
\section{Structure}
Structure objects can be created using the following methods:
\subsection{js methods}
\begin{list}{-}{}
 \item Neilos.Structure.new\_div: this will create a simple div. Additional classes can be added.
\item Neilos.Structure.new\_tab: this will create an entry (\bfseries{TODO} \normalfont Rename to new\_entry?). An entry is a div with 2 or 3 subdivs: [title], content, comments. Each subdivs has 2 subdivs: \_text and \_text\_right
\end{list}
\subsection{xml methods}
\begin{list}{-}{}
  \item \begin{minted}{xml}
<structure_div>         
        \end{minted}
create a div and a subdiv \_content. It is created at the end of the current DOM, so it should be used only in the main config. It must be placed inside $<$config$>$ tag.
\end{list}

\end{document}